\section{Interfacce}
 Le interfacce possono essere intese come un'evoluzione delle classi astratte e permettono, di fatto, di simulare l'ereditarietà multipla.
 \subsection{Pre Java 8}
 Sino a Java 7 un'interfaccia per definizione poteva possedere solo metodi dichiarati \textit{public} e \textit{abstract} e variabili dichiarate \textit{public}, \textit{static}, \textit{final}.
 Esempio di interfaccia:
 \begin{lstlisting}
public interface Saluto{
	public static final String CIAO = "Ciao";
	public static final String BUONGIORNO = "Buongiorno";

	public abstract void saluta();

}
 \end{lstlisting}
Ommettendo i modificatori, il programma funziona comunque in quanto vengono aggiunti automaticamente da Java. Viene aggiunto \textit{public abstract} per i metodi e \textit{public static final} per le variabili.

Non posso dichiarare i modificatori \textit{private, protected}.

\subsubsection{La dichiarazione di interfacce}
Un dichiarazione di interfaccia crea un nuovo tipo riferimento che avrà come istanze tutte le istanze delle classi che implementano l'interfaccia.

La definizione di un'interfaccia avviene tramite parla chiave \textbf{interface}. Un'interfaccia può contenere solo dichiarazioni di classi, interfacce, costanti e metodi astatti.
Si possono usare i seguenti modificatori:
\begin{itemize}
	\item \textbf{Abstract}: ogni interfaccia è implicitamente astratta;
	\item \textbf{Public}: tutti i membri di un'interfaccia sono dichiarati implicitamente come public;
	\item \textbf{Static}: le classi membro, le interfacce membro e i campi sono dichiarati implicitamente static;
	\item \textbf{Protected} e \textbf{Private}: si applica solo a interfacce membro a una dichiarazione di classe;
	\item \textbf{Final}: i campi sono implicitamente final, cioè definiscono delle costanti.
	\end{itemize}

Un'interfaccia non può estendere una classe ma può estendere una o più interfacce dette \textit{superinterfacce}. ereditando i suoi metodi.

La differenza tra implementare un'interfaccia ed estendere una classe, consiste nel fatto che possiamo estendere una sola classe alla volta. Possiamo invece implementare un numero indefinito di interfacce simulando l'ereditarietà multipla.

Riassumendo quindi una classe può implementare un'interfaccia e un'interfaccia può estendere un'altra interfaccia.

\subsection{Post Java 8}
\subsubsection{Metodi statici}
Dalla versione 8 di Java in poi è possibile definire all'interno delle interfacce anche \textbf{metodi statici}.
\begin{lstlisting}
public interface MyInterface{
	static void metodoStatico(){
	System.out.println("Metodo Statico Chiamato!");
	}
}
\end{lstlisting}
È possibile chiamare i metodi statici direttamente dalle interfacce:
\begin{lstlisting}
public interface TestMyInterface{
	public static void main (String args[]){
	MyInterface.metodoStatico();
	}
}
\end{lstlisting}
Un'interfaccia contenente metodi statici viola la natura costrattuale delle interfacce. Un'interfaccia simile non ha bisogno di essere implementata perchè i metodi statici di un'interfaccia non vengono ereditati. Interfacce di questo tipo serve solamente per definire funzioni.

\begin{lstlisting}
public class MyClass implements StaticMethodInterface{}

public class TestMyInterface{
	MyClass.metodoStatico(); /*errore: canno find symbol
	perchè il metodo statico non viene ereditato.*/

	StaticMethodInterface.metodoStatico(); /*ok, metodoStatico
	presente in StaticMethodInterface*/
}
\end{lstlisting}
Si possono creare anche dei metodi statici privati, chiamati solamente dalle interfacce che gli contengono. Agiscono da metodi ausiliari per i metodi statici.

\subsubsection{Metodi di default}
Un altra novità di Java 8 è la possibilità di dichiarare metodi concreti all'interno delle interfacce grazie alla keyword \textit{default}.
\begin{lstlisting}
interface TestInterface { 
	//square() è implicitamente public ed abstract
	void square(int a);
	default public void show(){
	System.out.print("i-test");
	} 
}

class Test implements TestInterface {
	public void square(int a) 
	{ ... }
	public static void main(...) {
	Test d = new Test();
	d.square(5); //ok, esegue square
	d.show();    //stampa: i-test
	}
}
\end{lstlisting}
In pratica è possibile aggiungere la segnatura \textit{default} davanti alla dichiarazione del metodo e dargli un'implementazione; così facendo tutte le classi che implementano quell'interfaccia avranno a disposizione quel metodo senza doverlo definire per forza.

Notare però che \textit{square(int a)} non è default e quindi, solito, serve definire il contratto nella classe che implementa l'interfaccia.Tuttavia il metodo di default "sparisce" quando viene ridefinito all’interno della classe che lo implementa:
\begin{lstlisting}
interface TestInterface { 
	//square() è implicitamente public ed abstract
	public void square(int a);
	default public void show(){
	System.out.print("i-test");
	} 
}

class Test implements TestInterface {
	public void square(int a) 
	{ ... }

	public void show(){
	System.out.print("i-test-over");
	}

	public static void main(...) {
	Test d = new Test();

	d.show();    //stampa: i-test-over
	}
}
\end{lstlisting}
Qui vediamo che è stato fatto l'override di show() (che è comunque un metodo virtuale!) e quindi la classe mostra i-test-over. in output. La regola è che la classe "vince sempre" perché se non appare nulla si usa il metodo default, altrimenti vale l'override.

Può capitare questo:
\begin{lstlisting}
interface TestInterface1{
 	default public void show(){
 	System.out.print("test1");
 	} 
 }

	 interface TestInterface2{ 
	 default public void show(){
	 System.out.print("test2");
 	} 
 }

 class Test implements TestInterface1, TestInterface2{
 	public void show(){
 		TestInterface1.super.show();
 		TestInterface2.super.show();
 	}
 }

 public static void main(...){
 ...
 d.show(); // stampa: i-test-over
 }
\end{lstlisting}

In questo caso c'è il problema che due interfacce diverse danno la stessa definizione del metodo show; il problema nasce perché Test implementa da entrambe le interfacce. Si crea un conflitto dato che Java non sa fra quale dei due scegliere. Per risolvere è necessario fare l'override e specificare cosa fare; in particolare la sintassi \textbf{NomeInterfaccia.super.metodo()} richiama il relativo metodo specificato di default.

La keyword default è stata introdotta per estendere le Collection API, sottoposta a vari cambiamenti, per aggiungere nuove funzionalità fornite da Java 8 quali lambda e Stream API, preservando la compatibilità con le librerie già esistenti.
\paragraph{Regole per l'utilizzo dei metodi default}
\begin{enumerate}
	\item Le classi vincono sulle interfacce. Quindi, se vi è una dichiarazione di un metodo concreto o astratto nelle catena dei sottotipi con la stessa segnatura di un metodo di default, possiamo ignorare le interfacce completamente. Valgono le stesse regole di invocazione dinamica già viste;
	\item Il figlio (sottotipo) vince sul padre (sovratipo). Se si presenta una situazione nella quale due interfacce forniscono un metodo di default con stessa segnatura e un interfaccia estende l'altra, allora il sottotipo vince.
	\item No rule3. Se applicate le prime due regole i conflitti non sono stati risolti, la sottoclasse dovrebbe (1) implementare il metodo oppure (2) dichiararlo astratto.
	\end{enumerate}
