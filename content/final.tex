\section{La keyword \textit{final}}
La keyword \textit{final} si può utilizzare in contesti diversi ma in generale identifica qualcosa di costante o immutabile.
Il reference this, che in un metodo non statico si riferisce all'oggetto di invocazione è sempre implicitamente \textit{final}, ovvero non modificabile. Invece l'oggetto a cui this si riferisce è modificabile:
\begin{lstlisting}
public class C {
	public int x;
	public void metodo(){
		this = new C(); /* ILLEGALE: cannot assign a 
		value to final variable this*/
		this.x = 10; //ok
	}
}
\end{lstlisting}

\subsection{Campi dati final}
Un campo dati statico o no marcato \textit{final} è un campo costante / immutabile, quindi ad ogni tentativo di modifica il compilatore segna un errore.
I campi dati marcati final devono essere sempre marcati esplicitamente, non basta l'inizializzazione automatica:
\begin{lstlisting}
public class E {
final int i; // ILLEGALE
}
public class E {
static final int k; // ILLEGALE
static final int v=0; //ok
}
public class E {
final int i;
E(){i=0;} // ok
}
public class E {
final int i;
E(){i=0; i=2;} /* ILLEGALE, non posso modificare
il valore di i dopo avergli assegnato 0*/
}
\end{lstlisting}
Se un riferimento \textit{ref} viene marcato \textit{final}, allora ref si riferisce ad uno ed un solo oggetto fissato, ma il contenuto dell'oggetto puntato può essere modificato. Java non fornisce un modo per rendere costanti gli oggetti:
\begin{lstlisting}
class Z{
	int i=2;
}

public class D{
	Z z1 = new Z();
	final Z z2 = new Z();
	static final Z z3 = new Z();
	final int[] a = {1, 2, 3, 4, 5};

	public static void main(String[] args){
	D d1 = new D();
	d1.z2.i++; //ok
	d1.z1 = new Z();
	for(int i = 0; i < d1.a.lenght; i++){
	d1.a[i]++; /* ok, solamente il reference a è costante
	d1.z2 = new Z(); ILLEGALE, z2 è final
	D.z3 = new Z(); ILLEGALE, z3 è final
	d1.a = new int[3]; ILLEGALE, a è final
	}
  } 
}
\end{lstlisting}

\subsection{Metodi \textit{final}}
Un metodo marcato come \textit{final} indica che tale metodo non può essere ridefinito in eventuali sottoclassi.
I motivi per marcare un metodo final sono:
\begin{itemize}
	\item \textbf{Sicurezza}: si può pensare che un metodo \textit{boolean validatePassword()} può essere marcato final per evitare evenutali ridefinizioni maliziose;
	\item \textbf{Efficienza};
	\item \textbf{Progettuali}: a volte si desidera mettere un blocco sul metodo per evitare che una sottoclasse cambi comportamento.
\end{itemize}

\begin{lstlisting}
public class Classe{
    //attributi
    public final void MetodoNonRidefinibile(){
        //codice del metodo non ridefinibile
    }
    //codice
}
\end{lstlisting}

\subsection{Classi \textit{final}}
Marcare una classe C come final significa che non possono essere definiti sottoclassi di C. Naturalmente, tutti i metodi di una classe final sono implicitamente final, mentre i campi dati possono o meno essere marcati final.
Si noti che marcare una classe final si opera una notevole restrizione a suo utilizzo in quanto non sarebbe permesso l'estendibilità di quella classe.