\section{La keyword \textit{abstract}}
\subsection{Metodi astratti}
È possibile definire dei metodi astratti all'interno delle classi astratte. Tali metodi, dovranno essere necessariamente ridefiniti (tramite override) nelle classi figlie. 

Un metodo astratto non può essere invocato in quanto non è definito, ma potrà essere soggetto a riscrittura( override ) in una sottoclasse.
\begin{lstlisting}
public abstract void Metodo();
\end{lstlisting}
\subsection{Le classi astratte}
In java una classe può essere dichiarata astratta usando il modificatore \textit{abstract}. 

Una classe astratta è una classe \textit{incompleta}, perchè contiene dei metodi astratti, cioè metodi d'istanza senza una corrispondente implementazione.

Data una astratta \textit{AbstractC} non posso creare istanze di quella classe tramite \textit{new AbstractC()}, perchè il loro comportamento non sarebbe determinato nel caso si invocasse su di esse un metodo astratto.
\begin{lstlisting}
public abstract class A {
double campoDati1;
/*una classe che contiene almeno un metodo astratto
dev'essere marcata abstract*/
	public abstract void metodo(); 
}
\end{lstlisting}
Una classe astratta può dichiarare dei campi (che ne descrivono lo stato) e dei metodi (non astratti) che ne specificano il funzionamento, oppure delle interfacce.

\textbf{Lo scopo} e l’utilità delle classi astratte è di gestire il comportamento di base delle classi che la derivano e che può essere ampliato e specializzato da queste ultime.
E' possibile avere dei costruttori astratti soprattutto quando ci sono dei campi dati nella classe,

Una \textit{classe concreta} che estende una classe astratta deve necessariamente implementare tutti i metodi astratti della superclasse, altrimenti il compilatore segnalerà un errore.

Se una classe concreta estende una classe abstract e in più è marcata abstract, allora tale sottoclasse è considerata una classe astratta.
Esempio:
\begin{lstlisting}
public abstract class A {
	private int a;
	public A(int x) {a=x;}
	public abstract int m(); 
	}

public class C extends A { // C e' una concretizzazione di A
	private int b;
	public int m() {return 4;} // implementazione di m()
	public C() {super(8); b=0;} // OK//
	public C(int x){}// ILLEGALE: No constructor matching A() 
	//found in class A
	public static void main(String args[]) { 
	 // A r = new A(9); //ILLEGALE: Abstract class A can't be 
	 //instantiated
	 A s = new C(); // OK
	 }  
	}
\end{lstlisting}

\subsection{Cose permesse vs cose non permesse}
\textbf{Non permesso}:
\begin{itemize}
	\item Non è possibile avere metodi statici astratti, altrimenti potrebbero essere invocati tramite il nome della classe senza istanziare un oggetto di quella classe astratta;
	\item Non è possibile avere un metodo finalastratto: sarebbe un \textit{no sense};
	\item Non si possono creare oggetti di classi astratte;
\end{itemize}

\textbf{Permesso}:
\begin{itemize}
	\item E’ possibile dichiarare un reference il cui tipo (statico) è una classe astratta A: il tipo dinamico di un tale reference dovrà quindi essere una sottoclasse concreta di A. Quindi una classe astratta può essere solamente un tipo statico, mai un tipo dinamico;
	\item E` possibile definire una classe astratta ma senza alcun metodo astratto. E’ anche possibile marcare astratta una classe che non definisce dei metodi (solo stato);
	\item In una classe astratta ci possono essere dei metodi implementati che invocano metodi astratti: a run-time verrà sempre invocato virtualmente l'opportuno metodo concreto
\end{itemize}
